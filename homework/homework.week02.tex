\documentclass[lang=cn,11pt,a4paper,cite=authoryear]{elegantpaper}

% 微分号
\newcommand{\dd}[1]{\mathrm{d}#1}
\newcommand{\pp}[1]{\partial{}#1}

\newcommand{\homep}[1]{\textbf{Problem #1}}
\newcommand{\subhomep}[1]{\textbf{SubProblem #1}}

% FT LT ZT
\newcommand{\ft}[1]{\mathscr{F}[#1]}
\newcommand{\fta}{\xrightarrow{\mathscr{F}}}
\newcommand{\lt}[1]{\mathscr{L}[#1]}
\newcommand{\lta}{\xrightarrow{\mathscr{L}}}
\newcommand{\zt}[1]{\mathscr{Z}[#1]}
\newcommand{\zta}{\xrightarrow{\mathscr{Z}}}

% 积分求和号

\newcommand{\dsum}{\displaystyle\sum}
\newcommand{\aint}{\int_{-\infty}^{+\infty}}

% 简易图片插入
\newcommand{\qfig}[3][nolabel]{
  \begin{figure}[!htb]
      \centering
      \includegraphics[width=0.4\textwidth]{#2}
      \caption{#3}
      \label{#1}
  \end{figure}
}

% 表格
\renewcommand\arraystretch{1.5}


% 日期


\title{数字电路基础\quad 第二周作业}
\author{范云潜 18373486}
\institute{微电子学院 184111 班}
\date{\zhtoday}

\begin{document}

\maketitle

作业内容:3.11,3.12,3.13; 3.3,3.4,3.7;

\homep{3.11}

对电压端进行戴维南等效:

\[V_{eq} = -10 + \frac{18}{18 + 5.1}(V_i + 10)\]

\[R_{eq} = 5.1 // 18 \approx 4 k\Omega\] 

输入为高 \(V_{eq,h} \approx 1.7V\), 为低\(V_{eq,l} \approx -2.2V\)

\[\begin{aligned}
    I_b &= \frac{1.7 - 0.7}{4 k} \approx 0.25 mA \\
    V_o &= 10 - 0.25 * 30 = 2.5 V > V_{CE,sat}
\end{aligned}\]

可以有效截止,但是不满足饱和条件,需要使 C 极电流变大,调节 \(V_{ih} = 6.7 V\) , \(V_{eq,h} \approx 3V\)

\[\begin{aligned}
    I_b &= \frac{3 - 0.7}{4 k} \approx 0.325 mA \\
    V_o &= 10 - 0.325 * 30 = 0.25 V > V_{CE,sat}
\end{aligned}\]

还可以调节 \(\beta\) 值或者减少 \(R_1\)

\homep{3.12}

饱和时:

\[I_b = \frac{5-0.1}{(2k)\cdot 50} = 4.9 \cdot 10^{-5} A\]

存在输入时,戴维南等效

\[R_{be} = 4.7 // 18 = 3.73 k\Omega\]

\(V = V_{ih} = 5V\) 时 

\[V_{eq} = 5 - 4.7 \frac{13}{22.7} = 2.3V\]

\[I_B = \frac{2.3 - 0.7}{22.7} = 0.42 mA\]

已经饱和, \(V_o = 0.1 V\)

\(V = V_{il} = 0V\) 时 \(V_{eq} < 0 \) 必然截止, \(V_o = 5V\)

输入悬空时 

\[V_{eq} = 5 - \frac{13}{25.7}7.7 = 1.1 V\] 

\[I_B = \frac{0.4}{3.73 K} = 0.11 mA\] 已经饱和,\(V_o = 0.1V\)

\homep{3.13}

通过拆分不同级进行计算:

a) 与级 + 倒向 + 偏压 + 倒向 + 输出 : \((A \cdot B)'' = A \cdot B\)

b) 输入 + 或非 + 偏压 + 倒向 + 输出 : \((A + B)'' = A + B\)

c) 输入 + 或非 + 输出 : \((A + B)'\)

d) 或非 + 三态反相器 : \(A' (G_1 + G_2)' + Z (G_1 + G_2)\)

\homep{3.3}

与非门: \(x' = (x \cdot 1)'\) 输入分别为变量与逻辑 1 ;
或非门: \(x' = (x + 0)'\) 输入分别为变量与逻辑0 ;
异或门: \(x' = (x \otimes 1)'\) 输入分别为变量与逻辑 1 。

\homep{3.4}

化简逻辑表达式: \(Y = (A'' \cdot B)' = (A' + B')\) ,考虑延时,如\figref{p1};不考虑,则恒为 1 。

\qfig[p1]{hw02p1.png}{结果}



\homep{3.7}


\subhomep{a} 

只关注 pMOS 一侧即可。\(Y = X'; X = (M' + N' + P'); M = A'; N = B'; P = C';\) 那么 \(Y = (A+B+C)'\) 。

\subhomep{b} 

只关注 pMOS 一侧即可。\(Y = X'' = X; X = (C'' B'' A'')\) 那么 \(Y = (A B C)\)

\subhomep{c}

以 A B 为输入的块输出为 \((A'+B')\) ;C D 输出为 \((C'+D')\) 。下一级输出 \((A'+B')' + (C'+D')' = (AB)+(CD)\) ,那么 \(Y = (INH' (AB+CD)'\)

\subhomep{d}

第一层输出 \(A'\) 和 \(B'\) ,\(A = 1, Y = B\) , \(B = 1, Y = A\) , \(B = A = 0, Y = Z\) 。那么 \(Y = AB + A'B'Z\)
% Start Here

% End Here

\end{document}