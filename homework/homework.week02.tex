\documentclass[lang=cn,11pt,a4paper,cite=authoryear]{elegantpaper}

\input{needed.tex}

\title{数字电路基础\quad 第二周作业}
\author{范云潜 18373486}
\institute{微电子学院 184111 班}
\date{\zhtoday}

\begin{document}

\maketitle

作业内容:3.11, 3.12,3.13

\homep{3.11}

对电压端进行戴维南等效:

\[V_{eq} = -10 + \frac{18}{18 + 5.1}(V_i + 10)\]

\[R_{eq} = 5.1 // 18 \approx 4 k\Omega\] 

输入为高 \(V_{eq,h} \approx 1.7V\), 为低\(V_{eq,l} \approx -2.2V\)

\[\begin{aligned}
    I_b &= \frac{1.7 - 0.7}{4 k} \approx 0.25 mA \\
    V_o &= 10 - 0.25 * 30 = 2.5 V > V_{CE,sat}
\end{aligned}\]

可以有效截止,但是不满足饱和条件,需要使 C 极电流变大,调节 \(V_{ih} = 6.7 V\) , \(V_{eq,h} \approx 3V\)

\[\begin{aligned}
    I_b &= \frac{3 - 0.7}{4 k} \approx 0.325 mA \\
    V_o &= 10 - 0.325 * 30 = 0.25 V > V_{CE,sat}
\end{aligned}\]

还可以调节 \(\beta\) 值或者减少 \(R_1\)

\homep{3.12}

饱和时:

\[I_b = \frac{5-0.1}{(2k)\cdot 50} = 4.9 \cdot 10^{-5} A\]

存在输入时,戴维南等效

\[R_{be} = 4.7 // 18 = 3.73 k\Omega\]

\(V = V_{ih} = 5V\) 时 

\[V_{eq} = 5 - 4.7 \frac{13}{22.7} = 2.3V\]

\[I_B = \frac{2.3 - 0.7}{22.7} = 0.42 mA\]

已经饱和, \(V_o = 0.1 V\)

\(V = V_{il} = 0V\) 时 \(V_{eq} < 0 \) 必然截止, \(V_o = 5V\)

输入悬空时 

\[V_{eq} = 5 - \frac{13}{25.7}7.7 = 1.1 V\] 

\[I_B = \frac{0.4}{3.73 K} = 0.11 mA\] 已经饱和,\(V_o = 0.1V\)

\homep{3.13}

通过拆分不同级进行计算:

a) 与级 + 倒向 + 偏压 + 倒向 + 输出 : \((A \cdot B)'' = A \cdot B\)

b) 输入 + 或非 + 偏压 + 倒向 + 输出 : \((A + B)'' = A + B\)

c) 输入 + 或非 + 输出 : \((A + B)'\)

d) 或非 + 三态反相器 : \(A' (G_1 + G_2)' + Z (G_1 + G_2)\)

% Start Here

% End Here

\end{document}