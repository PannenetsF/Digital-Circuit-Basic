\documentclass[lang=cn,11pt,a4paper,cite=authoryear]{elegantpaper}

\input{needed.tex}

\title{数字电路基础\quad 第八周作业}
\author{范云潜 18373486}
\institute{微电子学院 184111 班}
\date{\zhtoday}

\begin{document}

\maketitle

作业内容:6.5,6.7,6.35,6.10,6.12,6.19,6.22,6.29 

\homep{6.5}

两个 D 触发器都是下降沿触发的,分析其更新规则:

\(Q_{1,n+1}' = A Q_{2,n}'\) 

\(Q_{2,n+1}' = A (Q_{2,n}' Q_{1,n}')'\)

\(Y = A Q_1' Q_2\)

那么分析其变化

\begin{lstlisting}
A   Q1n'   Q2n'   Q1n+1'   Q2n+2'   Y 
0   x      x      1        1        0 -> 0 
1   0      0      1        0        0 -> 1
1   1      0      1        0        1 -> 1 
1   0      1      0        0        0 -> 0 
1   1      1      0        1        0 -> 0
\end{lstlisting}

状态图为 \figref{01}

\qfig[01]{h8.png}{状态图}

\homep{6.7}

首先,所有触发器的时钟沿一致,接下来分析更新规则:转换图如 \figref{02} 。

\(Q_{0,n+1} = Q_{0,n}', J = 1, K = 1\)

\(Q_{1,n+1} = (Q_{0,n}'(Q_{2,n}'Q_{3,n}')') \cdot Q_{1,n}' + Q_{0,n}Q_{1,n}, J = (Q_{0,n}'(Q_{2,n}'Q_{3,n}')'), K = Q_{0,n}'\)

\(Q_{2,n+1} = (Q_{3,n} Q_{0,n}') \cdot Q_{2,n}' + (Q_{1,n}' Q_{0,n}')' Q_{2,n}, J = (Q_{3,n} Q_{0,n}') , K = (Q_{1,n}' Q_{0,n}')\)

\(Q_{3,n+1} = (Q_{2,n}' Q_{1,n}' Q_{0,n}') \cdot Q_{3,n}' + Q_{0,n} Q_{3,n}, J =  (Q_{2,n}' Q_{1,n}' Q_{0,n}'), K = Q_{0,n}\)

\(Y = (Q_{0}' Q_{1}' Q_{2}' Q_{3}')\) 

% 又臭又长

\begin{lstlisting}
Q0  Q1  Q2  Q3  Q0n Q1n Q2n Q3n Y
0   0   0   0   1   0   0   1   1->0 % 0
1   0   0   1   0   0   0   1   0->0 % 9
0   0   0   1   1   1   1   0   0->0 % 1
1   1   1   0   0   1   1   0   0->0 % 14
0   1   1   0   1   0   1   0   0->0 % 6
1   0   1   0   0   0   1   0   0->0 % 10
0   0   1   0   1   1   0   0   0->0 % 2
1   1   0   0   0   1   0   0   0->0 % 12
0   1   0   0   1   0   0   0   0->0 % 4
1   0   0   0   0   0   0   0   0->1 % 8 
% 3 5 7 11 13 15
0   0   1   1   1   1   0   0   0->0
0   1   0   1   1   0   1   0   0->0
0   1   1   1   1   0   1   0   0->0
1   0   1   1   0   0   1   1   0->0
1   1   0   1   0   1   0   1   0->0
1   1   1   1   0   1   1   1   0->0
\end{lstlisting}

\qfig[02]{hw08p1.png}{6.7}

\homep{6.35}

转换图如 \figref{03} 。

\begin{lstlisting}
state_name  in      next        else
state_0     1       state_1     state_0
state_1     1       state_2     state_0
state_2     1       state_3     state_0
state_3     1       state_4     state_0 
state_4     1       state_4     state_0
\end{lstlisting}

\qfig[03]{hw08p2.png}{6.35}

\homep{6.10}

分析电路,加法器的和会更新到 \(A_3\) ,进位(低有效)会更新到 \(CI\) 。

\begin{lstlisting}
A3->A0  B3->B0  A3->A0{next}    CI{next}
1001    0011    0100            1
0100    0001    1010            0
1010    0000    0101            0
0101    0000    1010            0
\end{lstlisting}

可见,这是一个串行的加法器。

\homep{6.12}

\(EP = ET = 1\) 不会处于保持状态,\(R_D' = (Q_1Q_3)'\) ,为 \(1\) 时进行计数, \(0\) 时进行置零。状态如 \figref{04} 。

\begin{lstlisting}
Q3-Q0   RD' Q3-Q0{next} Y=Q3 
0000    1   0001        0->0
0001    1   0010        0->0
0010    1   0011        0->0
0011    1   0100        0->0
0100    1   0101        0->0
0101    1   0110        0->0
0110    1   0111        0->0
0111    1   1000        0->1
1000    1   1001        1->1
1001    1   1010        1->1
1010    0   0000        1->0
1011    0   0000        1->0
1100    1   1101        1->1
1101    1   1110        1->1
1110    0   0000        1->0
1111    0   0000        1->0
\end{lstlisting}

\qfig[04]{hw08p3.png}{6.12}

\homep{6.19}

分析两片计数器的接线方式。第一片的 \(EP, ET, LD, R_D\) 全部接高电平,一直处于计数状态,第二片的 \(EP\) 接到第一片的进位上,在第一片达到 \(1001\)  时触发第二片的计数。

在初始值分别为加载值 \(0000\)  和 \(0111\) 时,可知计数 20 个上升后,会得到 \(0000\) , \(1001\) 造成加载,得到 \(0000\) , \(0111\) ,因此这是 20 计数器。


\homep{6.22}

采用和上一题类似的思路,通过预置 0 进行工作。但是中间一级的进位不能持久作用,因此需要将 \(EP_3, ET_3\) 分开,让计数只持续一个时钟。如 \figref{05} 。

\qfig[05]{hw08p4.png}{6.22}

\homep{6.29}

利用 MUX 8 和 十进制计数器完成设计,将三位输出绑定到地址选择,另一位绑定到输入。列出变化需求,分析地址线:  

\begin{lstlisting}
Q3->Q0  OUT
0000    0
0001    0
0010    1
0011    0
0100    1
0101    1
0110    0
0111    1
1000    1
1001    1

ADDR = Q3-> Q1 
000 -> 0 
001 -> !Q0 
010 -> 1
011 -> Q0 
100 -> 1
\end{lstlisting}

\qfig[06]{hw08p5.png}{6.29}
\end{document}