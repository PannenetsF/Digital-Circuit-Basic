\documentclass[lang=cn,11pt,a4paper,cite=authoryear]{elegantpaper}

% 微分号
\newcommand{\dd}[1]{\mathrm{d}#1}
\newcommand{\pp}[1]{\partial{}#1}

\newcommand{\homep}[1]{\textbf{Problem #1}}
\newcommand{\subhomep}[1]{\textbf{SubProblem #1}}

% FT LT ZT
\newcommand{\ft}[1]{\mathscr{F}[#1]}
\newcommand{\fta}{\xrightarrow{\mathscr{F}}}
\newcommand{\lt}[1]{\mathscr{L}[#1]}
\newcommand{\lta}{\xrightarrow{\mathscr{L}}}
\newcommand{\zt}[1]{\mathscr{Z}[#1]}
\newcommand{\zta}{\xrightarrow{\mathscr{Z}}}

% 积分求和号

\newcommand{\dsum}{\displaystyle\sum}
\newcommand{\aint}{\int_{-\infty}^{+\infty}}

% 简易图片插入
\newcommand{\qfig}[3][nolabel]{
  \begin{figure}[!htb]
      \centering
      \includegraphics[width=0.4\textwidth]{#2}
      \caption{#3}
      \label{#1}
  \end{figure}
}

% 表格
\renewcommand\arraystretch{1.5}


% 日期


\title{数字电路基础\quad 第八周作业}
\author{范云潜 18373486}
\institute{微电子学院 184111 班}
\date{\zhtoday}

\begin{document}

\maketitle

作业内容:6.5,6.7,6.35,6.10,6.12,6.19,6.22,6.29 

\homep{6.5}

两个 D 触发器都是下降沿触发的,分析其更新规则:

\(Q_{1,n+1}' = A Q_{2,n}'\) 

\(Q_{2,n+1}' = A (Q_{2,n}' Q_{1,n}')'\)

\(Y = A Q_1' Q_2\)

那么分析其变化

\begin{lstlisting}
A   Q1n'   Q2n'   Q1n+1'   Q2n+2'   Y 
0   x      x      1        1        0 -> 0 
1   0      0      1        0        0 -> 1
1   1      0      1        0        1 -> 1 
1   0      1      0        0        0 -> 0 
1   1      1      0        1        0 -> 0
\end{lstlisting}

状态图为 \figref{01}

\qfig[01]{h8.png}{状态图}

\homep{6.7}

首先,所有触发器的时钟沿一致,接下来分析更新规则:转换图如 \figref{02} 。

\(Q_{0,n+1} = Q_{0,n}', J = 1, K = 1\)

\(Q_{1,n+1} = (Q_{0,n}'(Q_{2,n}'Q_{3,n}')') \cdot Q_{1,n}' + Q_{0,n}Q_{1,n}, J = (Q_{0,n}'(Q_{2,n}'Q_{3,n}')'), K = Q_{0,n}'\)

\(Q_{2,n+1} = (Q_{3,n} Q_{0,n}') \cdot Q_{2,n}' + (Q_{1,n}' Q_{0,n}')' Q_{2,n}, J = (Q_{3,n} Q_{0,n}') , K = (Q_{1,n}' Q_{0,n}')\)

\(Q_{3,n+1} = (Q_{2,n}' Q_{1,n}' Q_{0,n}') \cdot Q_{3,n}' + Q_{0,n} Q_{3,n}, J =  (Q_{2,n}' Q_{1,n}' Q_{0,n}'), K = Q_{0,n}\)

\(Y = (Q_{0}' Q_{1}' Q_{2}' Q_{3}')\) 

% 又臭又长

\begin{lstlisting}
Q0  Q1  Q2  Q3  Q0n Q1n Q2n Q3n Y
0   0   0   0   1   0   0   1   1->0 % 0
1   0   0   1   0   0   0   1   0->0 % 9
0   0   0   1   1   1   1   0   0->0 % 1
1   1   1   0   0   1   1   0   0->0 % 14
0   1   1   0   1   0   1   0   0->0 % 6
1   0   1   0   0   0   1   0   0->0 % 10
0   0   1   0   1   1   0   0   0->0 % 2
1   1   0   0   0   1   0   0   0->0 % 12
0   1   0   0   1   0   0   0   0->0 % 4
1   0   0   0   0   0   0   0   0->1 % 8 
% 3 5 7 11 13 15
0   0   1   1   1   1   0   0   0->0
0   1   0   1   1   0   1   0   0->0
0   1   1   1   1   0   1   0   0->0
1   0   1   1   0   0   1   1   0->0
1   1   0   1   0   1   0   1   0->0
1   1   1   1   0   1   1   1   0->0
\end{lstlisting}

\qfig[02]{hw08p1.png}{6.7}

\homep{6.35}

转换图如 \figref{03} 。

\begin{lstlisting}
state_name  in      next        else
state_0     1       state_1     state_0
state_1     1       state_2     state_0
state_2     1       state_3     state_0
state_3     1       state_4     state_0 
state_4     1       state_4     state_0
\end{lstlisting}

\qfig[03]{hw08p2.png}{6.35}


% Start Here

% End Here

\end{document}