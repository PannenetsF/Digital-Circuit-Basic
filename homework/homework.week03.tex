\documentclass[lang=cn,11pt,a4paper,cite=authoryear,twocolumn]{elegantpaper}

% 微分号
\newcommand{\dd}[1]{\mathrm{d}#1}
\newcommand{\pp}[1]{\partial{}#1}

\newcommand{\homep}[1]{\textbf{Problem #1}}
\newcommand{\subhomep}[1]{\textbf{SubProblem #1}}

% FT LT ZT
\newcommand{\ft}[1]{\mathscr{F}[#1]}
\newcommand{\fta}{\xrightarrow{\mathscr{F}}}
\newcommand{\lt}[1]{\mathscr{L}[#1]}
\newcommand{\lta}{\xrightarrow{\mathscr{L}}}
\newcommand{\zt}[1]{\mathscr{Z}[#1]}
\newcommand{\zta}{\xrightarrow{\mathscr{Z}}}

% 积分求和号

\newcommand{\dsum}{\displaystyle\sum}
\newcommand{\aint}{\int_{-\infty}^{+\infty}}

% 简易图片插入
\newcommand{\qfig}[3][nolabel]{
  \begin{figure}[!htb]
      \centering
      \includegraphics[width=0.4\textwidth]{#2}
      \caption{#3}
      \label{#1}
  \end{figure}
}

% 表格
\renewcommand\arraystretch{1.5}


% 日期


\title{数字电路基础\quad 第三周作业:张悦老师}
\author{范云潜 18373486}
\institute{微电子学院 184111 班}
\date{\zhtoday}

\begin{document}

\maketitle

作业内容:2.1-(2,6,7);
2.15-(2,5,8,9);
2.20-(c,d);
2.10-(2,4);
2.11-(4,5);
2.18-(6,7);
2.18-(8);
2.19-(2,5);
2.2-(2,3);
2.12-(1,3);
2.13-(2,3)
2.22-(1,4)

% \setlength{\columnseprule}{0.4pt}
% \begin{multicols*}{2}

\homep{2.1}

\subhomep{2}

\begin{table}[htb]
    \centering
    \caption{2.1.2}
    \label{tab:my-table}
    \begin{tabular}{ccc}
        \hline
        \(A\)         & 0 & 1 \\ \hline
        \(A\oplus 1\) & 1 & 0 \\
                    % &   &   \\ 
                    \hline
    \end{tabular}
\end{table}

\subhomep{6}

\begin{table}[htb]
    \centering
    \caption{2.1.6}
    \label{tab:my-table}
    \begin{tabular}{ccccc}
    \hline
    \(A\) & \(B\) & \(C\) & \(A(B\oplus C)\) & \(AB\oplus AC\) \\ \hline
    0     & 0     & 0     & 0                & 0               \\
    0     & 0     & 1     & 0                & 0               \\
    0     & 1     & 0     & 0                & 0               \\
    0     & 1     & 1     & 0                & 0               \\
    1     & 0     & 0     & 0                & 0               \\
    1     & 0     & 1     & 1                & 1               \\
    1     & 1     & 0     & 1                & 1               \\
    1     & 1     & 1     & 0                & 0               \\ \hline
    \end{tabular}
\end{table}

\subhomep{7}

\begin{table}[htb]
    \centering
    \caption{2.1.7}
    \label{tab:my-table}
    \begin{tabular}{cccc}
    \hline
    \(A\) & \(B\) & \((A\oplus B)'\) & \(A\oplus B\oplus 1\) \\ \hline
    0     & 0     & 1                & 1                     \\
    0     & 1     & 0                & 0                     \\
    1     & 0     & 0                & 0                     \\
    1     & 1     & 1                & 1                     \\ \hline
    \end{tabular}
\end{table}
% Start Here

\homep{2.15}

\subhomep{2}

\[
\begin{aligned}
    Y &= AB'C + A' + B + C' \\ 
    &= B'C + A' + B + C' \\
    &= C + A' + B + C' \\ 
    &= 1
\end{aligned}    
\]

\subhomep{5}

\[
\begin{aligned}
    Y &= AB'(A'CD + (AD + B'C')')(A'+B) \\
    &= (AB'A'CD + AB'((A'+D')(B+C))) \\
    & \quad\quad\quad\quad\quad (A'+B) \\ 
    &= AB'D'(B+C)(A'+B) \\
    &= AB'CD'(A'+B) \\ 
    &= 0
\end{aligned}    
\]

\subhomep{8}

\[\begin{aligned}
    Y &= A + (B+C')'(A+B'+C)(A+B+C) \\
    &= A + (B'C)(A+B'+C)(A+B+C) \\
    &= A + (B'C)(A+AB'+AC+\\
    &\quad\quad\quad\quad\quad AB+BC+AC+B'C+C) \\
    &= A + (B'C)(A+BC+B'C+C) \\ 
    &= A + B'C(A+C) \\ 
    &= A + AB'C + B'C \\ 
    &= A + B'C
\end{aligned}\]

\subhomep{9}

\[\begin{aligned}
    Y &= BC' + ABC'E + B'(A'D'+AD)' \\
    &\quad\quad\quad\quad\quad + B(AD'+A'D) \\
    &= BC' + B'(A+D)(A'+D') \\
    &\quad\quad\quad\quad\quad + B(AD'+A'D) \\ 
    &= BC' + B'(A'D+AD') + B(AD'+A'D) \\ 
    &= BC' + AD' + A'D
\end{aligned}\]

\homep{2.20}

\subhomep{c}

\[\begin{aligned}
    Y_1 &= ((AB')'(AD'C)')' \\
    &= ((A'+B)(A'+C'+D))' \\
    &= (A' + A'B + A'C' +BC' +A'D +BD)'\\
    &= (A' +BC' +BD)'\\
    &= A(BC')'(BD)' \\
    &= A(B'+C)(B'+D') \\
    &= (AB' + ACD')\\
    % 1000 1001 1010 1011 1010 1110
    % 8 9 10 11 10 14 
    &= \sum m(8,9,10,11,14) 
\end{aligned}\]

\[\begin{aligned}
    Y_2 &= ((AB')' (AC'D')' (A'C'D)' (ACD)')' \\
    &= (AB') + (AC'D') + (A'C'D) + (ACD) \\
    % 10xx, 1x00, 0x01, 1x11
    &= \sum m(1, 5, 8, 9, 10, 11, 12, 15)
\end{aligned}\]

\subhomep{d}

\[\begin{aligned}
Y_1 &= (AB) + (C(A\oplus B)) \\
&= AB + C(A'B + AB') \\
&= AB + A'BC + AB'C \\ 
% 110 111 011 101
&= \sum m(3, 5, 6, 7)
\end{aligned}\]

\[\begin{aligned}
Y_2 &= A\oplus B\oplus C \\
&(\text{真值为1时,输入为 001 111 101 011 })\\
&= \sum m(1, 3, 5, 7)
\end{aligned}\]

\homep{2.10}  实际上还是求与或形式

\subhomep{2}

\[\begin{aligned}
Y &= AB’C'D + BCD + A'D \\
% 1001 0111 1111 0001 0011 0101 0111
&= \sum m(1, 3, 5, 7, 9, 15)
\end{aligned}\]

\subhomep{4}

\[\begin{aligned}
Y &= AB + ((BC)'(C'+D'))' \\ 
&= AB + (BC + CD) \\ 
% 11xx x11x xx11
&= \sum m(3,  6, 7, 11,12, 13, 14, 15)
\end{aligned}\]

\homep{2.11}

\subhomep{4}

\[\begin{aligned}
Y &= BCD' + C + A'D \\
&= C + A'D \\
% C = 0 & A' = 0, D = 0
% 1x0x xx00 
&= \prod M(0, 4, 8, 9, 12, 13)
\end{aligned}\]

\subhomep{5}

\[
% \begin{aligned}
Y = \prod M(0, 3, 5)
% \end{aligned}    
\]

\homep{2.18}

\subhomep{6}

即在输入为 000, 001, 010, 101, 110, 111 时结果为 1 。建立卡诺图,如\figref{p1},结果为 \(A'B' + AC + BC'\)

\qfig[p1]{hw03p1.png}{2.18.6 卡诺图}

\subhomep{7}

即在输入为 0000, 0001, 0010, 0101, 1000, 1001, 1010, 1100, 1110 时结果为 1,建立卡诺图如\figref{p2},结果为 \(B'C' + AC'D' + A'C'D + ACD' + B'CD'\) 。


\qfig[p2]{hw03p2.png}{2.18.7 卡诺图}


\subhomep{8}

即在输入为 001,100,111 时结果为 1 ,建立卡诺图如\figref{p3} ,结果为\(A'B'C + AB'C' + ABC\)。

\qfig[p3]{hw03p3.png}{2.18.8 卡诺图}

\homep{2.19}

\subhomep{2}

\[\begin{aligned}
Y &= A'(CD'+C'D) + BC'D + AC'D\\
&\quad\quad\quad\quad\quad  + A'CD' \\ 
&= A'CD' + C'D
\end{aligned}\]

\subhomep{5}

\[\begin{aligned}
    Y &= (AB'C'D + AC'DE + B'DE' + \\
    &\quad\quad AC'D'E)' \\ 
    &= (AB'C'D + B'DE' + AC'E)' \\
    &= (AC'(B'D + E) + B'DE)' \\ 
    &= ((AC')' + (B'D)' E') ((B'D)' + E') \\
    &= (A'+C)(B+D') + (A'+C)E'\\
    &\quad\quad\quad\quad\quad  + (B+D')E' \\
    % &= A'B + BC + A'D' + CD' + A'E' + CE' \\
    % &\quad\quad\quad\quad\quad + BE' + D'E' \\
    &= A'B + A'D' + A'E' + BC + BE' \\
    &\quad\quad\quad\quad\quad + CD'  + CE' + D'E'
\end{aligned}\]

\homep{2.2}

\subhomep{2}

\[
\begin{aligned}
&(A+C')(B+D)(B+D') \\
=&(A+C')(B+BD+BD') \\ 
=&(A+C')B \\
=&AB+BC'
\end{aligned}    
\]

\subhomep{3}

\[\begin{aligned}
&((A+B+C')'C'D)'\\
& \quad\quad\quad\quad +(B+C')(AB'D+B'C')\\
&= (A'B'CC'D)' \\
& \quad\quad\quad\quad +(AB'C'D+B'C')\\
&= 1 +(AB'C'D+B'C') \\
&= 1 
\end{aligned}\]

\homep{2.12}

本题目的电路图通过 logisim 进行绘制

\subhomep{1}

如\figref{p4}

\[\begin{aligned}
Y &= (AB+BC+AC)'' \\
&= ((AB)' (BC)' (AC)')' \\
\end{aligned}\]

\qfig[p4]{hw03p4.png}{2.12.1 电路}

\subhomep{3}

如\figref{p5}

\[\begin{aligned}
    Y &= ((ABC')' (AB'C)' (A'BC)')''\\
    &= (((ABC')' (AB'C)' (A'BC)')' 1)'
\end{aligned}\]

\qfig[p5]{hw03p5.png}{2.12.3 电路}

\homep{2.13}

本题目的电路图通过 logisim 进行绘制

\subhomep{2}

如\figref{p6}

\[\begin{aligned}
    Y &= ((A+C)(A'+B+C')(A'+B'+C))'' \\
    &= ((A+C)' + (A'+B+C')' \\
    & \quad\quad + (A'+B'+C))' 
\end{aligned}\]

\qfig[p6]{hw03p6.png}{2.13.2 电路}
\subhomep{3}

如\figref{p7}

\[\begin{aligned}
    Y &= ((ABC' + B'C)'D' + A'B'D)'' \\
    &= (((ABC' + B'C)'D')' (A'B'D)' )'\\
    &= ((ABC'+B'C + D) (A+B+D'))' \\
    &= (ABC'+B'C + D)' + (A+B+D')' \\
    &= ((((A'+B'+C)' + (B+C')' + D)' \\
    & \quad\quad + (A+B+D')' )' + 0)' 
\end{aligned}\]

\qfig[p7]{hw03p7.png}{2.13.3 电路}
\homep{2.22}

\subhomep{1}

\[
\begin{aligned}
Y_1 &= AB'C'+ ABC + A'B'C + A'BC' \\
& \quad + A'B'C' + A'BC + A'B'C' + A'BC \\
&= B'C'+BC+A'B'+A'B
\end{aligned}    
\]

\subhomep{4}

\[
\begin{aligned}
Y_4 &= (AB'+B)CD' + ((A+B)(B'+C))' \\
&= AB'CD' + BCD' + A'B' + BC' \\
&= AB'CD' + BCD' + A'B' + BC' \\
& + ABCD' + ACD + BCD \\
&= AC + BC + A'B' + BC' \\
&= AC + B + A'B'
\end{aligned}    
\]

% End Here


% \end{multicols*}
\end{document}