\documentclass[lang=cn,11pt,a4paper,cite=authoryear, twocolumn]{elegantpaper}

\input{needed.tex}

\title{数字电路基础\quad 第七周作业}
\author{范云潜 18373486}
\institute{微电子学院 184111 班}
\date{\zhtoday}

\begin{document}

\maketitle

作业内容:5.14, 5.15, 5.16, 5.26

\homep{5.14}

只需考虑上升沿的状态,由于这系列时间内 \(R_D'\) 为高,只考虑 \(D\) 。如 \figref{1} 。

\qfig[1]{h7p1.png}{}


\homep{5.15}

JK 触发器规则为 \(Q^{n+1} = J Q^n {'} + K'Q^n\) ,那么在每个上升沿有:

\[\begin{aligned}
    init: Q &= 0\\
    clk0: Q &= 0 + 1 \& 0 = 0 \\
    clk1: Q &= 0 + 0 = 0 \\
    clk2: Q &= 1 \&  1 + x = 1 \\
    clk3: Q &= 0 + 0 = 0\\
    clk4: Q &= 1 
\end{aligned}\]

绘制如 \figref{2}

\qfig[2]{h7p2.png}{}

\homep{5.16}

T 触发器观察下降沿采样即可。设初值 Q 为 0 。绘制如 \figref{3}

\qfig[3]{h7p3.png}{}

\homep{5.26}

JK 端子均接高,那么 \(Q^{n+1} = J Q^n {'} + K'Q^n = Q^n {'}\) ,均接 \(Q_2\) 为 \(Q_3^{n+1} = Q_2 Q_3^n {'} + Q_2'Q_3^n = Q_2 \otimes Q_3^n\) 。但是需要注意到,实际上时钟有的来自之前信号的下降沿。

在每个 CLK 下降沿: 
\[\begin{aligned}
    init&: Q_1 = 0, Q_2 = 0, Q_3 = 0 \\
    clk0&: Q_1 = 1(Q_1' \downarrow), Q_2 = 1, Q_3 = 0 \\ 
    clk1&: Q_1 = 0(Q_1 \downarrow), Q_2 = 1, Q_3 = 1 \\ 
    clk2&: Q_1 = 1(Q_1' \downarrow), Q_2 = 0, Q_3 = 1 \\ 
    clk3&: Q_1 = 0(Q_1 \downarrow), Q_2 = 0, Q_3 = 1 \\ 
    clk4&: Q_1 = 1(Q_1' \downarrow), Q_2 = 1, Q_3 = 1 \\ 
    clk5&: Q_1 = 0(Q_1 \downarrow), Q_2 = 1, Q_3 = 0 \\ 
    clk6&: Q_1 = 1(Q_1' \downarrow), Q_2 = 0, Q_3 = 0 \\ 
    clk7&: Q_1 = 0(Q_1 \downarrow), Q_2 = 0, Q_3 = 0 \\
    &\text{Return to the init}\\
\end{aligned}\]

绘制如 \figref{4}

\qfig[4]{h7p4.png}{}
% Start Here

% End Here

\end{document}